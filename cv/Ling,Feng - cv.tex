\documentclass[10pt,letterpaper]{res}
\usepackage{color,soul}
\usepackage[latin1]{inputenc}
\usepackage{amsmath,amsfonts,amssymb}
\usepackage{array,longtable}
\usepackage{datetime}
\usepackage[left=.5in,right=1.25in,top=.5in,bottom=.5in]{geometry}

%xelatex parameter
\usepackage{xltxtra,fontspec,xunicode}
\defaultfontfeatures{Scale=MatchLowercase}
%\setromanfont{Adobe Garamond Pro}
\setromanfont{Garamond}
%xelatex parameter

%\usepackage{helvetica} % uses helvetica postscript font (download helvetica.sty)
%\usepackage{newcent}   % uses new century schoolbook postscript font

\newcommand{\tab}{\hspace*{2em}}
\newcommand{\vsrk}{\vspace*{-12pt}}
\newcommand{\sepline}{\smallskip\moveleft\hoffset\vbox{\hrule width 7.5in height .5pt}
\vspace*{-5pt}}

\newdateformat{monthyear}{%
  \monthname[\THEMONTH], \THEYEAR}

%\usepackage[compact]{titlesec}
%\titlespacing{\section}{0pt}{*0}{*0}
%\titlespacing{\subsection}{0pt}{*0}{*0}
%\titlespacing{\subsubsection}{0pt}{*0}{*0}

\begin{document}
\moveleft.5\hoffset\centerline{\huge\bf FENG LING}
\smallskip
\moveleft.5\hoffset\centerline{\monthyear\today}
\begin{resume}%
\vsrk
\section{PERSONAL}
\sepline
\begin{longtable}{@{}>{\raggedright}p{0.5\linewidth}
        p{\dimexpr0.5\linewidth-2\tabcolsep-\arrayrulewidth\relax}@{}}
{\bf Birth Year:} 1992 & {\bf Address:} 5505 Avenue F, Austin, TX 78751-1312\\
{\bf Citizenship:} China, People's Republic of & {\bf Mobile:} +1 (713) 666 - 2935\\
{\bf E-mail:} FLing@utexas.edu & {\bf Webpage:} http://fl3537.me/
\end{longtable}
\vsrk\vsrk
\section{EDUCATION}
\sepline
\begin{longtable}{@{}>{\raggedright}p{0.15\linewidth}
        p{\dimexpr0.85\linewidth-2\tabcolsep-\arrayrulewidth\relax}@{}}

{2010 - 2015} & {\bf The University of Texas at Austin}, Austin, TX\\
& { B.S. Pure Mathematics}, December 2015\\
& { B.S. Aerospace Engineering (Astronautics)}, December 2015\\
& { Computational Science and Engineering Certificate Program}, May 2015\\
& { Halliburton Business Foundations Summer Institute}, July 2012\\
& { GPA: 3.73/4.0 (188 GPA hr)}
\end{longtable}
\vsrk\vsrk
\section{EMPLOYMENT}
\sepline
\begin{longtable}{@{}>{\raggedright}p{0.15\linewidth}
        p{\dimexpr0.85\linewidth-2\tabcolsep-\arrayrulewidth\relax}@{}}
	{2013 - 2015} & {\bf Undergraduate Research Assistant}, Center for Space Research at UT Austin
	\\{2011} & {\bf Summer Intern}, Zhongchu Development Stock Ltd., Tianjin Xingang Branch
\end{longtable}
\vsrk\vsrk
\section{HONORS AND AWARDS}
\sepline
\begin{longtable}{@{}>{\raggedright}p{0.15\linewidth}
        p{\dimexpr0.85\linewidth-2\tabcolsep-\arrayrulewidth\relax}@{}}
	{2015} & {\bf Meritorious Winner Team Lead}, COMAP Mathematical Contest In Modeling\\
	& {\bf } Problem B: Searching lost aeroplane in open water, general advise from \textit{Dr~Andrew Spann}\\
	{2011} & {\bf Member}, Aerospace Honor Society Sigma-Gamma-Tau UT Austin Chapter \\
	{2010} & {\bf Team Lead}, Student Engineering Council Alternative Energy Challenge 3rd place \\
	{2010} & {\bf Finalist}, Intel International Science and Engineering Fair \\
%	{\bf 1st Place Team}, Greater Houston Area Science and Engineering Fair\\
%\tab	Harnessing Energy from Noises and Mechanical Vibrations\\
%	National AP Scholar, AP Scholar with Distinction\\
%	{\bf Graduate}, Bellaire Senior High School \hfill 2007-2010\\
%	{\bf Grade 6 out of 9}, Piano Specialty Certificate of Examination of China \hfill 2006\\
%	{\bf District 3rd place}, Tianjin Teenagers' Science Fair competition \hfill 2005\\
%\tab	Paper on oceanic energy and energy recycling from building waste water disposal system\\
%	{\bf Silver Medalist}, New Century Cup Contest of Juvenile Chinese Calligraphy \hfill 1999\\
\end{longtable}
\vsrk\vsrk
\section{PROJECTS}
\sepline
\begin{longtable}{@{}>{\raggedright}p{0.15\linewidth}
        p{\dimexpr0.85\linewidth-2\tabcolsep-\arrayrulewidth\relax}@{}}
	& {\bf 2D discrete inverse spectral problem}, supervised by \textit{Prof.~Etienne Vouga}\\
{2016 - now} &	Attempting to reconstruct an approximate solution surface given the discrete Laplace-Beltrami spectrum of a genus 0 surface\\

\end{longtable}
\vspace*{-1.5em}
\begin{longtable}{@{}>{\raggedright}p{0.15\linewidth}
        p{\dimexpr0.85\linewidth-2\tabcolsep-\arrayrulewidth\relax}@{}}
	& {\bf At Center for Space Research}, supervised by \textit{Prof.~Srinivas Bettadpur}\\
{2014 - 2015} &	Parametric study on dynamical effects of different misalignment models between spacecraft accelerometer and center of mass\\
{2014 - 2015} & Coding assists for GRACE spacecraft thermal environment modeling\\
{2014} & Analyzed correlations between GRACE accelerometer reading anomalies, thruster firing pattern, and star camera measurement deviations \\
{2013 - 2014} & Studied geographical significance of GRACE on-board SNR and post-fit residue of the Earth gravity model\\

\end{longtable}
\vspace*{-1.5em}
\begin{longtable}{@{}>{\raggedright}p{0.15\linewidth}
        p{\dimexpr0.85\linewidth-2\tabcolsep-\arrayrulewidth\relax}@{}}
{2014 - 2015} & 
{\bf For the CSE Certificate Program}, advised by \textit{Ren\'e Hiemstra}\\
& Investigated applications of discrete exterior calculus and discrete differential geometry for exact conservation finite element methods (mixed-methods)\\
& Explored some distributed computing implications using OpenMP\\

\end{longtable}
\vspace*{-1.5em}
\begin{longtable}{@{}>{\raggedright}p{0.15\linewidth}
        p{\dimexpr0.85\linewidth-2\tabcolsep-\arrayrulewidth\relax}@{}}
{2014} & {\bf Senior Design Project}, CubeSat Orbital Re-entry Vehicle System (CORVUS), in a team of 12\\
& Investigated challenges and possible solutions for the CubeSat orbital (LEO) re-entry problem \\
& In charge of simulation of the re-entry and parameter design for thermal subsystem\\

\end{longtable}
\vspace*{-1.5em}
\begin{longtable}{@{}>{\raggedright}p{0.15\linewidth}
        p{\dimexpr0.85\linewidth-2\tabcolsep-\arrayrulewidth\relax}@{}}
 & {\bf for Longhorn Rocket Association}\\
{2012 - 2014} & Designed and implemented software ground station and developed post-flight sensor fusion analysis for a high power (L2) rocket payload, joint with \textit{Scott Almond}\\
{2011} & Designed and machined model rockets from primitive components (e.g. uncured fiberglass)\\

\end{longtable}
\vspace*{-1.5em}
\begin{longtable}{@{}>{\raggedright}p{0.15\linewidth}
        p{\dimexpr0.85\linewidth-2\tabcolsep-\arrayrulewidth\relax}@{}}
{2012} & {\bf for Satellite Navigation Courses}, advised by Prof.~Todd Humphreys\\
	& Built a software GPS receiver based on Square Root Information Filters in MATLAB\\
	& Tested dual frequency carrier-phase differential GPS capability for the GRID receiver\\

\end{longtable}
\vspace*{-1.5em}
\begin{longtable}{@{}>{\raggedright}p{0.15\linewidth}
        p{\dimexpr0.85\linewidth-2\tabcolsep-\arrayrulewidth\relax}@{}}
{2010 - 2011} & {\bf TRICK Modeling and Simulation Research Initiatives}, in a team of 6\\
&	Generated Mars rover landing graphical simulation, results presented at NASA-JSC\\
&	Developed interfacing codes based on NASA software (TRICK, AGEA, and EDGE)
\end{longtable}
\vsrk\vsrk
\section{GRADUATE COURSEWORK}
\sepline
\begin{longtable}{@{}>{\raggedright}p{0.15\linewidth}
        p{\dimexpr0.85\linewidth-2\tabcolsep-\arrayrulewidth\relax}@{}}
{Spring 2016} & Kac-Moody Algebras and Groups (Auditing), \emph{Prof.~Daniel Allcock}\\
& Algebraic Geometry (Auditing), \emph{Prof.~David Ben-Zvi}\\
& Riemann Surfaces (Auditing), \emph{Prof.~Tim Perutz}\\
& Moduli of Higgs Bundle (Auditing), \emph{Prof.~Andrew Neitzke}\\

{Fall 2015} & Algebra, {\bf B}, \emph{Prof.~Felipe Voloch}\\
& K-theory as it appears in geometry, {\bf A}, \emph{Prof.~Dan Freed}\\
& 4-Manifold Topology (Audited), \emph{Prof.~Robert Gompf}\\
& Rational Homotopy Theory (Audited), \emph{Dr~Jonathan Campbell}\\

{Spring 2015} & Differential Topology, {\bf A-}, \emph{Prof.~Andrew Neitzke}\\
& D-modules (Audited), \emph{Dr~Sam Gunningham}\\
& Ergodic Theory and Dynamics (Audited), \emph{Prof.~Lewis Bowen}\\ 

{Fall 2014} & Real Analysis, {\bf A}, \emph{Prof.~Lewis Bowen}\\
& Algebraic Topology, {\bf B}, \emph{Prof.~Michael Starbird}\\
& Homotopy Type Theory (Audited), \emph{Prof.~Andrew Blumberg}\\

{Spring 2014} & Complex Analysis, {\bf A-}, \emph{Prof.~Thomas Chen}\\
& Stochastic Detection and Estimation, {\bf B+}, \emph{Prof.~Todd Humphreys}\\

{Fall 2013} & Finite Elements Methods, {\bf A}, \emph{Prof.~Mary Wheeler}\\

{Spring 2013} & GPS Signal Processing, {\bf A-}, \emph{Prof.~Todd Humphreys}\\
\end{longtable}
\vsrk\vsrk
\section{CONFERENCE COURSES}
\sepline
\begin{longtable}{@{}>{\raggedright}p{0.15\linewidth}
        p{\dimexpr0.85\linewidth-2\tabcolsep-\arrayrulewidth\relax}@{}}

{Fall 2015} & {\bf Topics in algebraic topology}, advised by \emph{Prof.~Andrew Blumberg}\\
& Mainly studying A Concise Course in Algebraic Topology (e.g. cup products (LS category), Poincar\'e duality, (co)fibrations and (co)fiber sequences, CW complex)
\end{longtable}
\vsrk\vsrk
\section{TALKS}
\sepline
\begin{longtable}{@{}>{\raggedright}p{0.15\linewidth}
        p{\dimexpr0.85\linewidth-2\tabcolsep-\arrayrulewidth\relax}@{}}
{Spring 2016} & {\bf Mathematics Undergraduate Student Talks (MUST)}, LS category and its cousins\\
{Fall 2015} & {\bf Directed Reading Program (DRP)}, (co)fiber sequences and $\pi_3(S^2)$, mentored by \textit{Ernest Fontes}\\
{Spring 2015} & {\bf DRP}, What is persistent homology, mentored by \textit{Ahmad Issa}\\
{Fall 2014} & {\bf DRP}, \v{C}ech cohomology of projective spaces, mentored by \textit{Dr~Yuecheng Zhu}\\
{Spring 2014} & {\bf DRP}, Classification of Du-val singularities, mentored by \textit{Dr~Yuecheng Zhu}\\
{Fall 2013} & {\bf DRP}, How to blow up double points in an affine plane and why you should do it too, mentored by \textit{Dr~Hendrik Orem}\\
\end{longtable}
\vsrk\vsrk
\section{MISC. EXTRACURRICULAR}
\sepline
\begin{longtable}{@{}>{\raggedright}p{0.15\linewidth}
        p{\dimexpr0.85\linewidth-2\tabcolsep-\arrayrulewidth\relax}@{}}
{2014 - 2016} & {\bf Participant}, TexTAG: Texas undergraduate Topology And Geometry conference\\
{2013 - 2016} & {\bf Active Member}, UT Undergraduate Math Club\\
{2011 - now} & {\bf Coursera, Udacity, and other MOOC experiences}\\
& Completed with Statement of Accomplishment in Cryptography, Software Testing, Machine Learning, Database Management, Artificial Intelligence, Automata Theory, Epigenetic Control of Gene Expression, Exploring Particle World, and Classical Chinese Philosophy.\\
{2011 - 2014} & {\bf Active Member}, Longhorn Rocket Association\\
{May 2014} & {\bf Participant}, LeaderShape Institute\\
{Summer 2013} & Programmed and assembled FPV-enabled quad-rotor PCB-frame MAV for fun\\
{2010 - 2011} & {\bf Active Member}, Engineering for a Sustainable World at UT Austin\\
{2010} & {\bf Member}, IEEE Robotics and Automation Society\\
& {\bf } Participated in Robot-a-thon autonomous robot building competition\\
% {2009 - 2010} & {\bf Sustainable Energy Team Project:}\\
% & Harnessing Energy from Noises and Mechanical Vibrations\\
% & Investigated actively for innovative solutions with limited resources\\
% &	Collaborated with generations of teammates for research and experiments\\
% &	Acquired skills on budget management and expanded knowledge on acoustics and materials\\
% &	Developed efficient presentation skills while requesting professional opinions and grants\\
% &	Improved time and stress management skills and scheduling\\
% {2010} & {\bf Active Member}, Freshman Engineering Committee of Student Engineering Council\\

\end{longtable}
\vsrk\vsrk
\section{VOLUNTEERING}
\sepline
\begin{longtable}{@{}>{\raggedright}p{0.15\linewidth}
        p{\dimexpr0.85\linewidth-2\tabcolsep-\arrayrulewidth\relax}@{}}

{2016} & SXSW (comedy and planning operations crew)\\
{2015} & Introduce a Girl to Engineering Day (Ballon rockets and iterative engineering design)\\
{Summer 2013} & UT Radionavigation Lab (Studying WAAS)\\
{2011} & Habitat for Humanity (Actually helped roofed and fenced a house) and Explore UT Guide\\
{2009} & Music Units Societies Everywhere (MUSE) and Bellaire Art Club\\
{2007 - 2009} & Methodist Hospital and Bellaire City Library\\
\end{longtable}
\end{resume} 
\end{document}